\documentclass[11pt, oneside]{article}   	% use "amsart" instead of "article" for AMSLaTeX format
\usepackage{geometry}                		% See geometry.pdf to learn the layout options. There are lots.
\geometry{letterpaper}                   		% ... or a4paper or a5paper or ... 
%\geometry{landscape}                		% Activate for rotated page geometry
%\usepackage[parfill]{parskip}    		% Activate to begin paragraphs with an empty line rather than an indent
\usepackage{graphicx}				% Use pdf, png, jpg, or eps§ with pdflatex; use eps in DVI mode
								% TeX will automatically convert eps --> pdf in pdflatex		
\usepackage{amssymb}
\usepackage{subeqnarray}
\newcommand{\bse}{\begin{subeqnarray}}
\newcommand{\ese}{\end{subeqnarray}}

%SetFonts

%SetFonts

\begin{document}

%\section{}
%\subsection{}

\centerline{{\bf Tutorial 2:  Lorenz-Reservoir Computing} }
\vspace{1cm}

The well-known toy model for atmospheric variability is the Lorenz 1963 model
given by the set of ODEs, 
% 
\bse
\frac{dx}{dt} &=& s (y - x) \\
\frac{dy}{dt} &=& r x - y - x z \\ 
\frac{dz}{dt} &=&  xy  - b z 
\ese
%
This system is, for  standard values of parameters $s=10, r=28$ and $b=2.667$,  
a  chaotic system with sensitivity to initial conditions. In this project, we will look 
at the skill of  Reservoir Computer  (RC) based predictions, using the Python
Notebook provided. 

%
\begin{itemize}
%
\item[(i)] Generate  data by using a time step $\Delta t = 0.02$ 
and integrating over the time interval $[-100,25]$. Use the interval 
$[-100,0]$ as training data and the interval $[0,25]$ as `truth'. 
%
\item[(ii)] Use initial  hyperparameter values $d = 300$ (reservoir size), 
$<k> = 6$ (mean degree), $\rho = 1.2$ (spectral radius), $\sigma = 0.1$ 
(interval in uniform distribution defining $W_{in}$), 
and $\beta = 0$ (regularization), see B5\_notes. Train 
the RC using the training data set and generate a RC prediction  
${\bf x}_R$ of the `truth'  ${\bf x}$ over the test interval $[0,25]$. 
Set a tolerance error 
%
\[
\epsilon =  || {\bf x} - {\bf x}_R ||^2
\]
%
and determine the time $t_e$ when $\epsilon > 0.1$. 
%
\item[(iii)]  How could one determine the Lyapunov exponents of the 
Lorenz system, using the RC approach (you do not have to do this, 
but you may)? 
% 
\item[(iv)] Study the effect of the hyperparameter $\rho$ on $t_e$
and try to explain the behavior.  
% 
\item[(v)] To which of the hyperparameters is $t_e$ most sensitive?  How 
could one optimize the values of these hyperparameters to maximize
$t_e$?  

\end{itemize}

\end{document}  